% Chapter 4

\chapter{Implementation and Evaluation} % Main chapter title

\label{Chapter4} % For referencing the chapter elsewhere, use \ref{Chapter1}

\lhead{Chapter 4. \emph{Implementation and Evaluation}} % This is for the header on each page - perhaps a shortened title

%----------------------------------------------------------------------------------------
\section{Development Stages }
Following are the stages of development:
\subsection{Strategy Stage}
We designed our web application while keeping in mind the requirements of our end sellers and customers. In order to make it a success we planned each and everything beforehand. We have learned our users demand and then planned our project on it. We have designed it in such a way that it can be easy to use and handle. After jotting down our requirements we made diagrams so that it can give an outlook of our system. After this we worked on our database and later on its implementation. We also wrote the project code, then we integrated and tested our it to verify if its working.

\section{Implementation}
Following is described about the implementation level things:

\subsection{Tools and Technologies}
\begin{enumerate}
	\item Server-Side API Framework \textbf{Ruby on Rails}
	\item Front-End Frameworks:
	\begin{itemize}
		\item \textbf{Next.js} (framework built on REACT Js library)
		\item \textbf{Bootstrap5} (CSS framework)
		\item \textbf{Google Material UI} (for interactive components)
		\item \textbf{Google Maps API}
	\end{itemize}
	\item DBMS \textbf{PostgreSQL}
	\item Version Control System (VCS): \textbf{Git, Github}
	\item Integrated Development Environment (IDE): \textbf{RubyMine, VS Code}
	\item API and Web testing tool: \textbf{Postman}
	\item Data Caching server: \textbf{Redis}
	\item Hosting Service for API: \textbf{Heroku}
	\item Hosting Service for Front-End App: \textbf{Amazon Static Hosting S3}
	\item CI | CD tool: \textbf{Github Actions}
	\item Media Storage Service: \textbf{AWS S3}
	\item Image Recognition Service: \textbf{AWS Rekognition}
	\item Domain Name Service: \textbf{AWS Route53}
	\item Domain Platform: \textbf{Namecheap.me}
	\item Bug Report Service: \textbf{Sentry}
	\item Mailer Service: \textbf{Mailjet}
	
\end{enumerate}


\subsection{Stage abc..}
Another example how to add reference \\
The concept is given in \citep{inth}


\section{System Integration }
\section{User Interface}
\section{Evaluation}
\section{Unit Testing}
\section{Functional Testing}
\subsection{Testing Requirements}
\section{Requirements}
\newpage
\begin{center}
 \textbf{ Some more samples of tables}
\end{center}
Another sample for Table \\ \\
\begin{tabular}{ |p{12cm}| }
 \hline
  \textbf{Algorithm 1:} Group-TS  \\ \hline
\textbf{ Input:} Transition system of each member \{$T_1$, $T_2$, $T_3$\}. \\
  \textbf{Output:} Combined Transition System of members that is \textbf{T}.
  \begin{enumerate}
    \item Let m = 1, 2, 3
    \item $q^0_T$ = $q^0_m$.
    \item recursive search on \textbf{T} starting from initial state ($q^0_T$).
    \item $q_m$ $\in$ q.
    \item The transition of $T_m$ is defined as $\rightarrow_m$. where $\rightarrow_m$ = (q,q')
    \item $\tau$ : set of possible transitions and $\rightarrow_m$ $\in$ $\tau$ then do.
    \item $\omega$ = minimum weight of $\rightarrow_m$ when the robot finds some region.
    \item Find new state of transition system that is q' and if q' not exists then.
    \item Insert the state q' to \textbf{T}.
    \item Insert new transition to $\delta_T$ with assigning the weight $\omega$. where $\delta_T$ is the set of transitions.
    \item prolong search from new state q'
    \item else if there is no transition for (q,q') then
    \item Include transition that is from (q,q') to $\delta_T$ with assigning the weight $\omega$
  \end{enumerate}
 \\ \hline
\end{tabular}

\vspace{20pt}

\begin{table}[h]
\centering
\caption{\textbf{(a)} Results after running the Transition System} % title of Table\addtolength{\tabcolsep}
\addtolength{\tabcolsep}{-2pt}
\begin{tabular}{|c|c|c|c|c|c|c|c|c|} % centered columns (9 columns)
\hline %inserts double horizontal lines
$\mathbb{T}$ & 0 & 2 & 3 & 4 & 6 & 8 & 10 & \dots \\  [2ex] % inserts table
%heading
\hline % inserts single horizontal line
$r*_{group}$ & $q_0$,$q_0$,$q_0$ & $q_1$,$q_1$,$q_1$ & $q_1$$q_0$1,$q_2$,$q_3$ & $q_0$,$q_1$,$q_2$ & $q_1$,$q_0$,$q_3$ & $q_0$,$q_1$,$q_2$ & $q_1$,$q_0$,$q_3$ & \dots \\ [2ex] % inserting body of the table
$L_T$(.) & . & $p_1$,$p_2$,$p_4$,$\sigma$ & $p_3$,$p_5$ & $p_2$,$p_6$,$\sigma$ & $p_1$,$p_5$,$\sigma$ & $p_2$,$p_6$,$\sigma$ & $p_1$,$p_5$,$\sigma$ & \dots \\ [1.5ex]
$r_1$* & $q_0$ & $q_1$ & . & $q_0$ & $q_1$ & $q_0$ & $q_1$ & \dots \\ [2ex]
$r_2$* & $q_0$ & $q_1$ & $q_2$ & $q_1$ & $q_0$ & $q_1$ & $q_0$ & \dots \\ [2ex]
$r_3$* & $q_0$ & $q_1$ & $q_3$ & $q_1$ & $q_0$ & $q_1$ & $q_0$ & \dots \\ [2ex] % [1ex] adds vertical space
\hline %inserts single line
\end{tabular}
\label{table:nonlin} % is used to refer this table in the text
\end{table}

We assume in Table 4.1 the first row shows the time when the transition occur, second row represents the run $r*_{group}$, third row corresponds to the satisfied propositions, and last three rows shows the separate run of these three robots. We observed in the optimal run that ($q_0$,$q_0$,$q_0$), ($q_1$,$q_1$,$q_1$), ($q_1$$q_0$1,$q_2$,$q_3$) is prefix and ($q_0$,$q_1$,$q_2$), ($q_1$,$q_0$,$q_3$) is suffix cycle and that will be repeated an infinite number of times. The given time which satisfies the $\sigma$ is $\mathbb{T^\sigma}$ = \{2,4,6,8,10,\dots\} and the function defined in (3.2) is $C(\mathbb{T})$ = 2.

Now, the time sequence when $\sigma$ is repeatedly satisfied is $\mathbb{T^\sigma}$ = \{2,4,6,8,10,\dots\} and cost function is given as;

 \hspace{30pt}  $C(\mathbb{T})$ = $\lim_{k\to+\infty}$ ($\mathbb{T}$(k+1) - $\mathbb{T}$(k))

\hspace{30pt}  = $\mathbb{T}$(k+1) - $\mathbb{T}$(k)

\hspace{30pt}  = 4 - 2 = 2

 As $C(\mathbb{T})$ = 2 is the obtained function where the $\sigma$ is repeatedly satisfied. At t=3, robot2 has reached at $q_2$ while the robot1 is still moving from $q_1$ to $q_0$, therefore $r_1$* has no correlated state to t=3.

 \subsection{Accurate Specified Run}
 In Algorithm.2 the exact solution is summarized and it shows that a particular solution is given to a specified problem for that case where robots have exact time information.

\begin{center}
\begin{tabular}{ |p{12cm}| }
 \hline
 \textbf{Algorithm 2:} Accurate-Run \\ \hline
\textbf{ Input:} \{$T_1$,$T_2$, $T_3$\} \& LTL formula $\phi$. \\
  \textbf{Result:} Different runs of each system in the form \{$r_1$*,$r_2$*,$r_3$*\} that satisfies $\phi$.
  \begin{enumerate}
    \item n = 1, 2, 3
    \item Model the group transition system \textbf{T}.
    \item Now search runs $r*_{group}$ for the system as done in Table.1.
    \item Trace runs $r*_{group}$ on the transition systems $T_n$ to obtain the runs $r_n$*.
    \item Then find obtained function where $\sigma$ is satisfied.
  \end{enumerate}
 \\
 \hline
\end{tabular}
\end{center}

As the grouped transition system \textbf{T} is constructed by using Algorithm.1 to model the team. After that we obtain a run $r*_{group}$ on \textbf{T} that satisfies the LTL formula.

\vspace{10pt}
\begin{table}[h]
\centering
\caption{\textbf{(b)} Results after running the Transition System} % title of Table
\addtolength{\tabcolsep}{-6pt}
\small
\begin{tabular}{|c|c|c|c|c|c|c|c|c|c|c|} % centered columns (9 columns)
\hline %inserts double horizontal lines
$\mathbb{T}$ & 0 & 2 & 3 & 4 & 5 & 6 & 7 & 8 & 9 & \dots \\  [2ex] % inserts table
%heading
\hline % inserts single horizontal line
$r*_{group}$ & $q_0$,$q_0$,$q_0$ & $q_1$,$q_1$,$q_1$ & $q_1$$q_0$1,$q_2$,$q_3$ & $q_0$,$q_1$,$q_2$ & $q_0$$q_1$1,$q_2$,$q_3$ & $q_1$,$q_1$,$q_1$ & $q_1$$q_0$1,$q_2$,$q_3$ & $q_0$,$q_1$,$q_2$ & $q_0$$q_1$1,$q_2$,$q_3$ & \dots \\ [2ex] % inserting body of the table
$L_T$(.) & . & $p_1$,$p_2$,$p_4$,$\sigma$ & $p_3$,$p_5$ & $p_2$,$p_6$,$\sigma$ & $p_3$,$p_5$ & $p_1$,$p_2$,$p_4$,$\sigma$ & $p_3$,$p_5$ & $p_2$,$p_6$,$\sigma$ & $p_3$,$p_5$ & \dots \\ [2ex]
$r_1$* & $q_0$ & $q_1$ & . & $q_0$ & . & $q_1$ & . & $q_0$ & . & \dots \\ [2ex]
$r_2$* & $q_0$ & $q_1$ & $q_2$ & $q_1$ & $q_2$ & $q_1$ & $q_2$ & $q_1$ & $q_2$ & \dots \\ [2ex]
$r_3$* & $q_0$ & $q_1$ & $q_3$ & $q_1$ & $q_3$ & $q_1$ & $q_3$ & $q_1$ & $q_3$ & \dots \\ [2ex] % [1ex] adds vertical space
\hline %inserts single line
\end{tabular}
\label{table:nonlin} % is used to refer this table in the text
\end{table}

\vspace{10pt}
Now similarly in Table 4.2 the first row shows the time when the transition occur, second row represents the run $r*_{group}$, third row corresponds to the satisfied propositions, and last three rows shows the separate run of three robots. As we observed in the optimal run that ($q_0$,$q_0$,$q_0$), ($q_1$,$q_1$,$q_1$) is the prefix and ($q_1$$q_0$1,$q_2$,$q_3$), ($q_0$,$q_1$,$q_2$), ($q_0$$q_1$1,$q_2$,$q_3$), ($q_1$,$q_1$,$q_1$) is the suffix cycle and that will be repeated an infinite number of times. The given time which satisfies the $\sigma$ is $\mathbb{T^\sigma}$ = \{2,4,6,8,10,\dots\} and the function defined in (3.2) is $C(\mathbb{T})$ = 2.

Now as we done in Table 4.1 similarly the same procedure for this run shown in Table 4.2, the time sequence when $\sigma$ is repeatedly satisfied is $\mathbb{T^\sigma}$ = \{2,4,6,8,10,\dots\} and cost function is given as;

\hspace{30pt}  $C(\mathbb{T})$ = $\lim_{k\to+\infty}$ ($\mathbb{T}$(k+1) - $\mathbb{T}$(k))

\hspace{30pt}  = $\mathbb{T}$(k+1) - $\mathbb{T}$(k)

\hspace{30pt}  = 4 - 2 = 2

 As $C(\mathbb{T})$ = 2 is the obtained function where the $\sigma$ is repeatedly satisfied.

 For some applications including new states and corresponding transitions to the structure of the robotic system may indicate to introducing advance stages or motion commands at some lower level. So the proper way in which the changes of these models are strictly application specific and we do not consider such details in our work. Assuming that these changes can be implemented in future.

 \subsection{Synchronized Specification}
 If there is a situation in which robots are moving at uncertain time and $\phi$ is not satisfied then we consider individual synchronization run for the robots that helps in guarantee the correctness of the model. In Algorithm.3 the protocols which is followed by the robots for synchronization in the field.

 \begin{center}
\begin{tabular}{ |p{13cm}| }
 \hline
 \textbf{Algorithm 3:} Synchronized-Move \\ \hline
\textbf{ Input:} $r_i$ and sequence $s_i$ of $robot_i$ \\
  \textbf{Result:} Simple synchronized moves for each robot that satisfies $\phi$.
  \begin{itemize}
    \item Initialized z at 0 and while True do
  \end{itemize}
    \begin{enumerate}
    \item report all the members.
    \item wait until all members received their information message.
    \item after satisfying propositions at $r_i^{z}$ create a transition to $r_i^{z+1}$.
    \item z = z+1.
  \end{enumerate}
 \\
 \hline
\end{tabular}
\end{center}


